\section{Introduction}
%-------------------------------------------------------------------------------
Strongly consistent distributed databases (DSDBs) provide scalable storage capacities for large scale web applications without sacrificing easy-to-reason-about concurrency semantics. However, they do not achieve good throughput under skewed workloads, even though real-world workloads are inherently skewed. On the other hand, single machine databases (SMDBs) better accommodate skew in their workloads due to their throughputs being higher by several orders of magnitude. In exchange, their storage capacities cannot scale past a single machine.

When searching for a database to support skewed workloads, users are forced to choose between performant SMDBs or scalable DSDBs. Skew creates “hotkeys” which are highly contended for. In a distributed setting, hotkeys are a product of uneven load concentrated on a few keys that ultimately leave some nodes overwhelmed but others underutilized. In addition, contention induces cycles of deadlocks, wait queues, retries, and aborts that the database from efficiently utilizing nodes’ processing power to execute transactions.

A DSDB’s throughput can easily scale with the number of nodes in the case of uniform or partitionable workloads via load balancing techniques like sharding, but these techniques are poorly suited for the case of highly skewed workloads. Hotkeys inundate their host nodes (“hotshards”) beyond their maximum throughputs such that requests cannot be processed fast enough to clear a growing queue. The hotshards cannot clear their large workload fast enough because they are not sufficiently provisioned--either in terms of hardware or software--to sustain such high throughput. Without provisioning very powerful hotshards, the hotshards themselves become a bottleneck for system throughput.

SMDBs accommodate skewed workloads better than their distributed counterparts because they sustain fundamentally higher throughput. The primary reason for this is that SMDBs need not communicate over a network. While an RPC between two nodes in the same datacenter costs at least a few milliseconds, a message between two CPU cores costs at most a few microseconds. Thus, distributed transactions take several orders of magnitude longer to complete. The communication cost decreases throughput by decreasing the number of transactions that the database can eke out per unit time simply because transactions have a higher latency. A DSDB’s communication cost is a fundamental bottleneck that prevents it from achieving the same throughput as SMDBs.

Furthermore, the high latency of distributed transactions generates more contention on popular keys at lower skews. Contention is a serious bottleneck in strongly consistent systems that prevents systems from utilizing their nodes’ throughput capability for executing transactions; no useful work is done when transactions expend node resources waiting, retrying, or aborting instead of committing. These systems often implement concurrency control protocols such as 2PL or OCC that lock keys for significant durations of time, making them more prone to contention than their weakly consistent counterparts. Skew induces contention, because it creates a small set of hotkeys that a large proportion of requests will attempt to lock. Only one transaction can lock any given key at a time, so high contention forces transactions to serialize their execution on hotkeys. This effectively reduces the system’s throughput to the processing rates of individual nodes. In a distributed setting, transactions’ higher latencies prolong the durations for which the transactions hold locks, exacerbating the conflict. At best, this results in head-of-line blocking. More likely, following transactions that fail to acquire their locks will repeatedly retry, deadlock, and/or abort, all of which leads to decreased system throughput.

We introduce Thermopylae, a distributed database that offers throughput on the order of magnitude of a SMDB. Thermopylae introduces a novel architecture that embeds a SMDB into a DSDB. We exploit the inherent skew in real world workloads and co-locate hotkeys on a single hotshard to create a central point to which we can apply targeted optimizations. We then completely replace the hotshard with a SMDB instance, which we fully integrate into the overall DSDB. The novel architecture and co-location together prevent the hotshard from becoming a bottleneck by applying the SMDB’s superior throughput to the workload’s hottest keys. In order to alleviate contention and enable Thermopylae to access the full throughput capability of our novel architecture, we design a complementary concurrency control protocol, 3PL, that reduces contention on the hotshard and coordinates two distinctly executing databases to preserve strict serializability. 

\rewrite{\textbf{<Here is a paragraph reserved for eval numbers.>} Lorem ipsum lorem ipsum text text text.}

In summary, this paper makes the following contributions:
\begin{itemize}
    \item A novel distributed architecture that increases throughput \true{by several orders of magnitude}.
    \item An complementary concurrency control protocol that enforces strict serializability by coordinating two independent databases.
    \item \true {Thermopylae, a distributed database capable of sustaining throughput around three orders of magnitude higher than existing state of the art.}
\end{itemize}